\documentclass{ctexart}
\usepackage[left=1in, right=1in, top=1.2in, bottom=1.2in]{geometry}
\usepackage{ctex}
\usepackage[utf8]{inputenc}
\usepackage{fontspec}
\usepackage{a4wide}
% \setmainfont[Scale = 1]{Palatino}
% \setCJKmainfont{Songti SC}
\usepackage{fancyhdr}
\usepackage{amsmath}
\usepackage{stmaryrd}
\usepackage{setspace}
\pagestyle{fancy}
\fancypagestyle{plain}{
    \fancyhead[L]{East China Normal University}
    \fancyhead[R]{}
    \fancyfoot[C]{\thepage}
}
\def\premise{\mathrm{premise}}
\def\assumption{\mathrm{assumption}}
\def\MT{\mathrm{MT\ }}
\def\LEM{\mathrm{LEM}}
\def\PBC{\mathrm{PBC\ }}
\def\intro{\mathrm{i\ }}
\def\elim{\mathrm{e\ }}
\def\introa{\mathrm{i_1\ }}
\def\elima{\mathrm{e_1\ }}
\def\introb{\mathrm{i_2\ }}
\def\elimb{\mathrm{e_2\ }}

\def\n{\neg}
\def\d{\vee}
\def\c{\wedge}
\def\a{\forall}
\def\e{\exists}

\def\assignment{\textsf{Assignment}}
\def\implied{\textsf{Implied}}
\def\ifstmt{\textsf{If-Statement}}
\def\invarianthg{\textsf{Invariant Hyp.} \c \textsf{Guard}}
\def\partialwhile{\textsf{Partial-While}}

\newcommand{\hoare}[1]{\qquad\llparenthesis\ #1 \ \rrparenthesis}

\def\max{\text{max}}
\def\ran{\text{ran}}

\title{Logic in Computer Science Assignment 6}
\author{10185101210 陈俊潼}
\date{January 2021}

\begin{document}

\maketitle

\section{Prove}

\subsection{Prove the following progam \texttt{MAX} is partial correctness.}
\begin{equation*}
    \begin{split}
        \{n > 0 \c dom(f) =& [0\ldots n-1] \c ran(f) \in N\} \\
        & Max \\
        \{y = &max\Big(ran(f)\Big)\}
    \end{split}
\end{equation*}

\textbf{Proof:}

First we add an empty else statement and a pair of brace for the original code to make it more clear in the process of proving. So the refined version of the \texttt{MAX} program would be:

\begin{flalign*}
    \linespread{1}
    & \texttt{y = f[0]} \\
    & \texttt{i = 1} \\
    & \texttt{while (i < n)} \\
    & \texttt{\{} \\
    & \qquad \texttt{if (y < f[i])} \\
    & \qquad \texttt{\{} \\
    & \qquad \qquad \texttt{y = f[i])} \\
    & \qquad \texttt{\}} \\
    & \qquad \texttt{else} \\
    & \qquad \texttt{\{} \\
    & \qquad \texttt{\}} \\
    & \qquad \texttt{i = i + 1} \\
    & \texttt{\}}
\end{flalign*}
\newpage
Then analysing the algorithm, we assume the loop invariant to be \texttt{y = max(ran(f[0:i]))}. Now we can proof its partial correctness as follows:
\setlength{\jot}{1pt}
\begin{flalign*}
    \setstretch{0}
    & \hoare{n > 0 \c dom(f) = [0\ldots n-1] \c ran(f) \in N} \\
    & \hoare{\top}  & \implied \\
    & \hoare{f[0] = \max(f[0])} & \implied \\
    & \hoare{f[0] = \max(\ran(f[0:1]))} & \implied \\
    & \texttt{y = f[0]} \\
    & \hoare{y = \max(\ran(f[0:1]))} & \assignment \\
    & \texttt{i = 1} \\
    & \hoare{y = \max(\ran(f[0:i]))} & \assignment \\
    & \texttt{while (i < n)} \\
    & \texttt{\{} \\
    & \qquad \hoare{y = \max(\ran(f[0:i])) \c i< n} & \invarianthg \\
    & \qquad \hoare{y = \max(\ran(f[0:i]))} & \implied \\
    & \qquad \hoare{\Big(y<f[i] \to \max(\ran(f[0:i])) < f[i] \Big) \c \\ & \qquad \qquad \qquad \qquad \qquad \n(y<f[i]) \to y = \max(\ran(f[0:i]))} & \implied \\
    & \qquad \texttt{if (y < f[i])} \\
    & \qquad \texttt{\{} \\
    & \qquad \qquad \hoare{\max(\ran(f[0:i])) < f[i]} & \ifstmt \\
    & \qquad \qquad \hoare{f[i] = \max(\ran(f[0:i]),\ f[i])} & \implied \\
    & \qquad \qquad \texttt{y = f[i]} \\
    & \qquad \qquad \hoare{y = \max(\ran(f[0:i]),\ f[i])} & \assignment \\
    & \qquad \qquad \hoare{y = \max(\ran(f[0:i+1]))} & \implied \\
    & \qquad \texttt{\}} \\
    & \qquad \texttt{else} \\
    & \qquad \texttt{\{} \\
    & \qquad \qquad  \hoare{y = \max(\ran(f[0:i]))} & \ifstmt \\
    & \qquad \qquad  \hoare{y = \max(\ran(f[0:i+1]))} & \implied \\
    & \qquad \texttt{\}} \\
    & \qquad \hoare{y = \max(\ran(f[0:i+1]))} & \ifstmt \\
    & \qquad \texttt{i = i + 1} \\
    & \qquad \hoare{y = \max(\ran(f[0:i]))} & \assignment \\
    & \texttt{\}} \\
    & \hoare{y = \max(\ran(f[0:i])) \c \n(i < n)} & \partialwhile \\
    & \hoare{y = \max(\ran(f[0:i])) \c i \ge n } & \implied \\
    & \hoare{y = \max(\ran(f[0:n]))} & \implied \\
    & \hoare{y = \max(\ran(f))} & \implied
\end{flalign*}

\end{document}
